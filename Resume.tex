%Document Class Definition
\documentclass{article}

%Required Packages
\usepackage{xcolor}
\usepackage{xstring}
\usepackage{fontawesome}
\usepackage{tikz}
	\usetikzlibrary{shapes.geometric}
	\usetikzlibrary{calc}
	\usetikzlibrary{decorations.pathreplacing}
\usepackage[utf8]{inputenc}
\usepackage[T1]{fontenc}
\usepackage[left=1cm,right=1cm,top=1cm,bottom=1cm]{geometry}
\usepackage{setspace}
\usepackage[absolute,overlay]{textpos}
\usepackage{hyperref}
\usepackage{tabularx}
\usepackage{calc}
\usepackage{ifthen}
\usepackage{coolstr}
\usepackage{textcase}
\usepackage{amsmath}
\usepackage{relsize}
\usepackage{moresize}
\usepackage{times}
\usepackage{enumitem}
\usepackage{booktabs}
\usepackage{multirow}
\usepackage{colortbl}
\usepackage{mathabx}

%Definition of the colors
\definecolor{white}{RGB}{255,255,255}
\definecolor{sidecolor}{RGB}{200,200,200}
\definecolor{mainblue}{RGB}{55,55,135}
\definecolor{maingray}{RGB}{144,144,144}

%Define CV Variables
\newcommand{\cvfirstname}{Stéphane}
\newcommand{\cvlastname}{Caron}
\newcommand{\cvtitle}{B. Sc Actuariat}
\newcommand{\cvbirthdate}{1er Octobre 1992}
\newcommand{\cvemail}{ste.caron@icloud.com}
\newcommand{\cvphonenumber}{(581) 997-5006}
\newcommand{\cvaddress}{113 des traversiers,\newline Lévis}
\newcommand{\cvlocation}{Quebec, Canada}
\newcommand{\cvlinkedin}{https://www.linkedin.com/in/stéphane-caron-45828267/}
\newcommand{\cvwhitewords}{Agile,Applications,ASP.NET,Build,C\#,Code,Collaborate,Data,Data Scientist,
Design,Development,Engineering,Environment,Java,Javascript,Knowledge,Machine Learning,Mobile,.NET,Product,
Projects,Software,Solutions,SQL,Team,Technologies,Testing,Tools,Web}

%Command for vertical centering of text into tabular
\renewcommand{\tabularxcolumn}[1]{m{#1}}

%Command for horizontal centering of specified width into tabular
\newcolumntype{M}[1]{>{\centering\arraybackslash}m{#1}}

%Command for bold font and textcolor to apply to a specific column into tabular
\newcolumntype{A}[1]{>{\bfseries\centering\color{mainblue}}m{#1}}

%Command to print checkmark with desired format into Certificate Table
\newcommand{\CheckMark}{\textcolor{mainblue}{\faCheck}}

%Command to create the correct seperator for Interests
\newcommand{\separator}{\textcolor{mainblue}{ $\blackdiamond$ }}

%Command to produce a round shaped corner framebox with defined colors
\newcommand{\mybox}[1]{
	\tikz
	\node[rectangle,rounded corners=1mm,draw=mainblue,anchor=text,fill=mainblue,inner sep=3pt,minimum height=0.5cm]
		{\small\bfseries\textcolor{white}{#1}};}

%Command to create the sidebar layout
\newcommand{\sidebar}{
	\begin{tikzpicture}[overlay]
		\node [rectangle, fill=sidecolor, anchor=north west, minimum width=9.25cm, 
		minimum height=\paperheight+2cm] (box) at (-2cm,3cm){};%
	\end{tikzpicture}}

%Command for printing the margin section titles
\newcommand{\profilesection}[1]{
	\vspace{6pt}
	{\bfseries\Large #1\;\rule[0.15\baselineskip]{7cm-\widthof{\bfseries\Large #1\;}}{1pt}\normalfont\normalsize}
	\vspace{6pt}}

%Command for printing the icons
\newcommand{\icon}[1]{
	\begin{tikzpicture}[baseline=(char.base)]
		\node[circle,draw,baseline=(char.base),mainblue,fill=mainblue,text=white,
		inner sep=0.25pt,minimum size=10mm,align=center,text centered](char){\large\textbf{{#1}}};
	\end{tikzpicture}}

%Command to create the star scale
\newcommand\starscale[2]{
	\pgfmathsetmacro\pgfxa{#1+1}
	\tikzstyle{scorestars}=[star,star points=5,star point ratio=2,thick,mainblue,scale=1.5,draw,inner sep=0.125em,anchor=outer 	point 3]
	\begin{tikzpicture}[baseline]
		\foreach \i in {1,...,#2} {
			\pgfmathparse{(\i<=#1?"mainblue":"maingray")}
			\edef\starcolor{\pgfmathresult}
			\draw (\i*1em,0) node[name=star\i,scorestars,fill=\starcolor]{};}
		\pgfmathparse{(#1>int(#1)?int(#1+1):0}
	\let\partstar=\pgfmathresult
	\ifnum\partstar>0
		\pgfmathsetmacro\starpart{#1-(int(#1))}
		\path [clip] ($(star\partstar.outer point 3)!(star\partstar.outer point 2)!(star\partstar.outer point 4)$) 
		rectangle 
		($(star\partstar.outer point 2 |- star\partstar.outer point 1)!\starpart!(star\partstar.outer point 1 -| star\partstar.outer point 5)$);
		\fill (\partstar*1em,0) node[scorestars,fill=mainblue]{};
	\fi
	\end{tikzpicture}}

%Command for printing skill progress bars
\newcommand{\progressbarthickness}{0.1}
\newcommand\skillbar[1]{
	\begin{tikzpicture}
		\foreach [count=\i] \x/\y in {#1}{
			\draw[fill=maingray,maingray] (0,\i) rectangle (7,\i+\progressbarthickness);
			\draw[fill=white,mainblue](0,\i) rectangle (\y,\i+\progressbarthickness);
			\node[circle,fill,mainblue] at (\y,\i+\progressbarthickness/2){};
			\node[circle,fill,maingray,scale=0.5] at (\y,\i+\progressbarthickness/2){};
			\node[above right] at (0,\i+\progressbarthickness){\x};}
	\end{tikzpicture}}

%Command for printing white words
\newcommand\printww[1]{
	\tiny\textcolor{white}{
	\foreach \x in #1{\x \- \,}}}

%Command to create section Title
\newcommand\sectiontitle[1]{
	\bfseries\LARGE
	\StrLen{#1}[\longueur]
	\textcolor{mainblue}{\StrMid{#1}{1}{3}}%
	\textcolor{black}{\StrMid{#1}{4}{\longueur}}%
	\normalfont\normalsize}

%Command to let mark in order to trace tikz picture
\newcommand{\tikzmark}[1]{\tikz[overlay,remember picture] \node[circle,scale=0.01] (#1) {};}

%Command to insert a new Education Statement
\newcommand{\EduStatement}[6]{
	\normalfont\normalsize
	\vspace{-\baselineskip}
	\tikzmark{mark1}
	\vspace{12pt}\\
	#4\\
	\large{\textbf{#5}}\\
	\textcolor{mainblue}{#3\small\textbf{\hfill [#1 - #2]}}\\
	\normalfont\normalsize
	\tikzmark{mark2}
	}

%Command to insert a new Experience Statement
\newenvironment{ExpStatement}[4]{
	\normalfont\normalsize
	\vspace{-\baselineskip}
	\tikzmark{mark1}
	\vspace{12pt}\\
	#3\\
	\large{\textbf{#4}}\\
	\small
	\textcolor{mainblue}{\textbf{Réalisations \hfill [#1 - #2]}}
	\begin{itemize}[label=\textcolor{mainblue}{\textbullet},topsep=0pt,partopsep=0pt,itemsep=0pt,parsep=0pt]}
		{\vspace{-\baselineskip}
	\end{itemize}
	\normalfont\normalsize
	\tikzmark{mark2}
	\\}

\pagestyle{empty} % Disable headers and footers
\setlength{\parindent}{0pt} 
\setlength{\TPHorizModule}{1cm}
\setlength{\TPVertModule}{1cm}
\hypersetup{
colorlinks,
linkcolor={red!50!black},
citecolor={blue!50!black},
urlcolor={blue!80!black}}

%Edition of the content
\begin{document}
\sidebar
\begin{textblock}{7}(0.5,1)
	\HUGE
	\cvfirstname\\[6pt]
	\bfseries
	\cvlastname\\[6pt]
	\large
	\cvtitle\\
\end{textblock}

\begin{textblock}{7}(0.5, 3.75)
	\profilesection{Informations générales}	
	\vspace{-\baselineskip}
	\begin{table}
		\begin{tabularx}{7cm}{cX}
			\icon{\faBirthdayCake}&\cvbirthdate\\
			\icon{\faAt}&\bfseries\href{mailto:\cvemail}\cvemail\\
			\icon{\faPhone}&\cvphonenumber\\
			\icon{\faHome}&\cvaddress\\
			\icon{\faMapMarker}&\cvlocation\\
			\icon{\faLinkedin}&\href{\cvlinkedin}{See Profile}\\
		\end{tabularx}
	\end{table}

	\profilesection{Langue}
	\vspace{-\baselineskip}
	\begin{table}
		\large
		\begin{tabularx}{7cm}{Xc}
			Français & \starscale{4.5}{5}\\
			Anglais & \starscale{3.5}{5}\\
		\end{tabularx}
	\end{table}

	\profilesection{Connaissances en programmation}
	\skillbar{Git \faGithub/3.5,\LaTeX/3,R/6,Python/2.5,VBA/5.5,SAS/5.5,SQL/5}
\textbf{En cours d'apprentissage:} \\
Python et programmation GPU
\end{textblock}

\begin{textblock}{11.35}(9.25, 0.25)
	\printww{\cvwhitewords}
\end{textblock}

\begin{textblock}{11.35}(9.25, 1)
  \sectiontitle{Éducation}
	\EduStatement{09/2017}{Present}{Université Laval}{Maîtrise}{Statistiques}\\
	\vspace{-10pt}
	\EduStatement{09/2012}{05/2015}{Université Laval}{Baccalauréat}{Actuariat}
	\\
	\sectiontitle{Expérience}
	\begin{ExpStatement}{05/2015}{Présent}{Promutuel Assurance}{Analyste en actuariat}
		\item Développement de rapports dynamiques \texttt{RMarkdown} et \texttt{shiny} pour suivre les résultats d'assurance.
		\item Création de plusieurs programmes servant à extraire et valider les données de conduite du programme de télématique.
		\item Plusieurs travaux en lien avec le développement et l'intégration du produit télématique.
	\end{ExpStatement}
	\\
	\sectiontitle{Certificats/Examens}
	\vspace{3pt}
	\begin{table}
		\begin{tabularx}{\textwidth}{A{1.5cm}XM{1.5cm}}
			\toprule
			\centering\textcolor{black}{Exam}&\centering\textbf{Description}&\textbf{Date}\\
			\midrule
			P/1&Probability&01/2013\\
			FM/2&Financial Mathematics&06/2013\\
			MFE/3F&Models for Financial Economics&11/2013\\
			MLC/3LC&Models for Life Contingencies&10/2014\\
			3ST&Models for Stochastic Processes and Statistics&12/2014\\
			C/4&Construction and Evaluation of Actuarial Models&02/2014\\
			CA1&CAS Risk Management and Insurance Operations&09/2014\\
			CA2&CAS Insurance Accounting, Coverage Analysis, Insurance Law, and Insurance Regulation&01/2015\\
			ECON&VEE Economics&08/2015\\
			CORPFIN&VEE Corporate Finance&08/2015\\
			APPSTATS&VEE Applied Statistics&08/2015\\
			EXAM 5&Basic Techniques for Ratemaking and Estimating Claim Liabilities&12/2015\\
			COP&Course on Professionalism&12/2016\\
			\bottomrule
 		\end{tabularx}
 	\end{table}
 	\vspace{\baselineskip}
	\sectiontitle{Working\ Skills\ \&\ Strengths}\\
	\mybox{Modelisation} \mybox{Business, Risk and Cost-Benefit Analysis} \mybox{Creative} \mybox{Passionate}\\
	\mybox{Willis Towers Watson: Pricing Softwares} \mybox{Hard-Worker} \mybox{Validation Process}\\
	\mybox{Business Optimization} \mybox{Critical-Thinking} \mybox{User Experience} \mybox{Quick Leaner}\\
	\newline
	\sectiontitle{Achievements}\\
	Participation to Academic Actuarial Competitions - ActuLab
	\begin{itemize}[label=\textcolor{mainblue}{\textbullet},topsep=0pt,partopsep=0pt,itemsep=0pt,
	parsep=0pt]
		\item Extrapolation of Truncated Claim Amounts\\
		\small\bfseries\textcolor{mainblue}{
		{1\textsuperscript{st} Place - Desjardins, General Insurance Group \hfill 2015}}
		\normalsize\normalfont
		\item Asset-Liability Matching: Optimization under Constraints\\
		\small\bfseries\textcolor{mainblue}{
		1\textsuperscript{st} Place - SSQ, Financial Group \hfill 2015\\
		Presentation of the solution at the Cercle Finance Quebec}
		\normalsize\normalfont
		\item Territorial Grouping for Ontario PAP under FISCO Constraints\\
		\small\bfseries\textcolor{mainblue}{
		5\textsuperscript{th} Place - Desjardins, General Insurance Group \hfill 2014}
		\normalsize\normalfont
	\end{itemize}
	Obtaining a Scholarship of Excellence\\
	\small\bfseries\textcolor{mainblue}{University Laval \hfill 2011}\\
	\normalfont\normalsize\\
	\sectiontitle{Interests}\\
	Physics \separator Reading \separator Fitness \separator Cosom Hockey \separator Machine Learning \separator Philanthropy
\end{textblock}
\end{document}